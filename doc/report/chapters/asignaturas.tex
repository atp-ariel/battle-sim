\chapter{Reflejo de las asignaturas en el proyecto}

\section{Simulaci\'on}

Tendr\'iamos como sistema el enfrentamiento b\'elico. Las entidades ser\'ian las unidades, estructuras y elementos del terreno. Como relaciones tendr\'iamos por ejemplo la distancia entre estas entidades, el daño que le causa una unidad a otra, etc. Como proceso tendr\'iamos el movimiento de las unidades, el ataque de una unidad a otra,etc.

Este sistema es observable, permiti\'endonos, al ejecutar simulaciones del mismo, obtener resultados y sacar conclusiones a partir de estos. Es controlable pues las unidades realizan acciones seg\'un estrategias y la simulaci\'on ocurre seg\'un reglas definidas. Es modificable pues podemos agregar y eliminar unidades, adem\'as de cambiar reglas y estrategias, lo que nos permite obtener diferentes resultados.

\section{Compilaci\'on}

Se definir\'a un lenguaje en el cual se puedan definir diferentes unidades, estructuras y sus respectivas estad\'isticas, modificar las caracter\'isticas del terreno, crear estrategias y definir reglas para la simulaci\'on.

\section{Inteligencia Artificial}
Como las unidades se tendr\'an que mover por el mapa, para lograr un movimiento eficiente de las mismas utilizaremos el algoritmo A*.

Adem\'as como tenemos dos bandos enfrent\'andose nos auxiliaremos de un algoritmo Minimax para realizar los movimientos que har\'an los bandos en sus respectivos turnos, utilizando una heur\'istica basada en la situaci\'on actual del mapa y la estrategia definida por el usuario.

Es muy posible que con el avance de la asignatura y el desarrollo del proyecto utilicemos m\'as herramientas.