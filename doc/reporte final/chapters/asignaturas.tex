\section{Reflejo de las asignaturas en el proyecto}

\subsection{Simulación}

Tendríamos como sistema el enfrentamiento bélico. Las entidades serían las unidades, estructuras y elementos del terreno. Como relaciones tendríamos por ejemplo la distancia entre estas entidades, el daño que le causa una unidad a otra, etc. Como proceso tendríamos el movimiento de las unidades, el ataque de una unidad a otra,etc.

Este sistema es observable, permitiéndonos, al ejecutar simulaciones del mismo, obtener resultados y sacar conclusiones a partir de estos. Es controlable pues las unidades realizan acciones según estrategias y la simulación ocurre según reglas definidas. Es modificable pues podemos agregar y eliminar unidades, además de cambiar reglas y estrategias, lo que nos permite obtener diferentes resultados.

\subsection{Compilación}

Se definirá un lenguaje en el cual se puedan definir diferentes unidades, estructuras y sus respectivas estadísticas, modificar las características del terreno, crear estrategias y definir reglas para la simulación.

\subsection{Inteligencia Artificial}
Como las unidades se tendrán que mover por el mapa, para lograr un movimiento eficiente de las mismas utilizaremos el algoritmo A*.

Además como tenemos dos bandos enfrentándose nos auxiliaremos de un algoritmo Minimax para realizar los movimientos que harán los bandos en sus respectivos turnos, utilizando una heurística basada en la situación actual del mapa y la estrategia definida por el usuario.

Es muy posible que con el avance de la asignatura y el desarrollo del proyecto utilicemos más herramientas.